\documentclass[10pt,a4paper]{article}
\usepackage[utf8]{inputenc}
\usepackage{amsmath}
\usepackage{amsfonts}
\usepackage{amssymb}

\begin{document}
\textbf{Les codes
}
\section{Provider de login}
Creation du provider AuthProvider pour l'authentification
\begin{verbatim}
import React, { createContext, useState, useEffect, ReactNode } from 'react';


const initialValues = {
    email: "",
    token: "",
    authorities: "",
  };
export const AuthContext = createContext();

export const AuthProvider = ({ children }) => {
  const [user, setUser] = useState(initialValues);

  return (
    <AuthContext.Provider value={{ user, setUser }}>
      {children}
    </AuthContext.Provider>
  );
};

\end{verbatim}
\section{Le composant login pour l'authentification}
\begin{verbatim}
import React, { useEffect } from 'react';
import {Formik, Field, Form, ErrorMessage} from 'formik';
import * as Yup from 'yup';
import "bootstrap/dist/css/bootstrap.css";
import { AuthContext } from '@/context/AuthProvider';

  


const validationSchema = Yup.object().shape({
   
 
    email: Yup.string()
        .email("email invalide")
        .required("l'email est obligatoire"),
    password: Yup.string()
        .required("Mot de passe est obligatoire")
        .min(4, "Mot de passe doit être plus grand que 8 caractères")
        .max(50, "Mot de passe doit être plus petit que 50 caractères"),
});




interface DetailsProps {
    cours: {
        id:number,
    },
  }

  const Login =  ({ navigation }) => {
    const initialValues = {
        email: "",
        password: "",
    };

    const { user , setUser } = React.useContext(AuthContext);
  
const handleSubmit = async (values) => {
    //console.log(cours.id);
    
    
    try {
        const response = await fetch('http://localhost:8080/api/login', {
            method: 'POST',
            headers: {
                'Content-Type': 'application/json',
            },
            body: JSON.stringify(values),
        });

        if (!response.ok) {
          //  throw new Error('Une erreur est survenue lors de l\'enregistrement des données.');
            alert("mot de passe ou email incorrect");
        }

        const data = await response.json();
        console.log('Données enregistrées:', data);
        setUser(data);
        if(user){
            navigation.navigate('cours');
        }
    }
    
    catch (error) {
        console.error('Erreur:', error);
        // Optionally handle errors appropriately, e.g., show a notification
    }
}

useEffect (() => {  
    console.log(user);
    
}, [user]);

    return (
        <div className="container">
            <div className="row">
                <div className="col-md-6 offset-md-3 pt-3">
                    <Formik
                        initialValues={initialValues}
                        validationSchema={validationSchema}
                        onSubmit={handleSubmit}
                        >
                        {({ resetForm }) => (
                            <Form>
                                
                                <div className="form-group mb-3">
                                    <label htmlFor="email">
                                        email:
                                    </label>
                                    <Field
                                        type="email"
                                        id="email"
                                        name="email"
                                        className="form-control"
                                    />
                                    <ErrorMessage
                                        name="email"
                                        component="small"
                                        className="text-danger"
                                    />
                                </div>
                              
                                <div className="form-group mb-3">
                                    <label htmlFor="password">
                                        Mot de passe:
                                    </label>
                                    <Field
                                        type="password"
                                        id="password"
                                        name="password"
                                        className="form-control"
                                    />
                                    <ErrorMessage
                                        name="password"
                                        component="small"
                                        className="text-danger"
                                    />
                                </div>
                               
                               
                                <div className="form-group d-flex justify-content-end gap-3">
                                    <button
                                        type="submit"
                                        className="btn btn-primary"
                                    >
                                        Se connecter
                                    </button>
                                    <button
                                        type="button"
                                        onClick={resetForm}
                                        className="btn btn-danger"
                                    >
                                        Annuler
                                    </button>
                                </div>
                            </Form>
                        )}
                    </Formik>
                </div>
            </div>
        </div>
    );
};   
export default Login;

\end{verbatim}

\section{Register}
\begin{verbatim}
import React from 'react';
import {Formik, Field, Form, ErrorMessage} from 'formik';
import * as Yup from 'yup';
import "bootstrap/dist/css/bootstrap.css";
import {  useRouter, } from 'expo-router';  // Importez useRouter
const router = useRouter();
const validationSchema = Yup.object().shape({
    nom: Yup.string()
        .min(3, "trop petit")
        .max(255, "trop long!")
        .required("Ce champ est obligatoire"),
    prenom: Yup.string()
        .min(3, "trop petit")
        .max(255, "trop long!")
        .required("Ce champ est obligatoire"),
   
    sexe: Yup.string()
    .min(2, "trop petit")
    .max(10, "trop long!")
    .required("Ce champ est obligatoire"),
    role: Yup.string()
    .min(2, "trop petit")
    .max(10, "trop long!")
    .required("Ce champ est obligatoire"),
 
    email: Yup.string()
        .email("email invalide")
        .required("l'email est obligatoire"),
    password: Yup.string()
        .required("Mot de passe est obligatoire")
        .min(8, "Mot de passe doit être plus grand que 8 caractères")
        .max(50, "Mot de passe doit être plus petit que 50 caractères"),
});




interface DetailsProps {
  cours: {
      id:number,

     
  
  },
}
const Register = ({ navigation }) => {
  const initialValues = {
    nom: "",
    prenom: "",
    email: "",
    sexe: "",
    role: "user",

    password: "",
};

        const handleSubmit = async (values) => {
            try {
                const response = await fetch('http://localhost:8080/api/users', {
                    method: 'POST',
                    headers: {
                        'Content-Type': 'application/json',
                    },
                    body: JSON.stringify(values),
                });

                if (!response.ok) {
                    throw new Error('Une erreur est survenue lors de l\'enregistrement des données.');
                }

                const data = await response.json();
                console.log('Données enregistrées:', data);
               if(data){
                navigation.navigate('login');
            }
               
            }
            
            catch (error) {
                console.error('Erreur:', error);
                
            }
        }

    return (
        <div className="container">
            <div className="row">
                <div className="col-md-6 offset-md-3 pt-3">
                    <h1 className="text-center">Créer un compte</h1>
                    <Formik
                        initialValues={initialValues}
                        validationSchema={validationSchema}
                        onSubmit={(values) =>handleSubmit(values)}
                    >
                        {({ resetForm }) => (
                            <Form>
                                <div className="form-group mb-3">
                                    <label htmlFor="nom">
                                        nom:
                                    </label>
                                    <Field
                                        type="text"
                                        id="nom"
                                        name="nom"
                                        className="form-control"
                                    />
                                    <ErrorMessage
                                        name="nom"
                                        component="small"
                                        className="text-danger"
                                    />
                                </div>
                                <div className="form-group mb-3">
                                    <label htmlFor="prenom">
                                        prenom:
                                    </label>
                                    <Field
                                        type="text"
                                        id="prenom"
                                        name="prenom"
                                        className="form-control"
                                    />
                                    <ErrorMessage
                                        name="prenom"
                                        component="small"
                                        className="text-danger"
                                    />
                                </div>
                                
                                <div className="form-group mb-3">
                                    <label htmlFor="email">
                                        email:
                                    </label>
                                    <Field
                                        type="email"
                                        id="email"
                                        name="email"
                                        className="form-control"
                                    />
                                    <ErrorMessage
                                        name="email"
                                        component="small"
                                        className="text-danger"
                                    />
                                </div>
                                <div className="form-group mb-3">
                                        <label htmlFor="sexe">Sexe:</label>
                                        <div role="group" aria-labelledby="sexe">
                                            <label>
                                                <Field type="radio" name="sexe" value="masculin" className="form-check-input" />
                                                Masculin
                                            </label>
                                            <label>
                                                <Field type="radio" name="sexe" value="feminin" className="form-check-input" />
                                                Féminin
                                            </label>
                                        </div><ErrorMessage name="sexe" component="small" className="text-danger" />
                                </div>
                                <div className="form-group mb-3">
                                    <label htmlFor="password">
                                        Mot de passe:
                                    </label>
                                    <Field
                                        type="password"
                                        id="password"
                                        name="password"
                                        className="form-control"
                                    />
                                    <ErrorMessage
                                        name="password"
                                        component="small"
                                        className="text-danger"
                                    />
                                </div>
                               
                               
                                <div className="form-group d-flex justify-content-end gap-3">
                                    <button
                                        type="submit"
                                        className="btn btn-primary"
                                    >
                                        S'inscrire
                                    </button>
                                    <button
                                        type="button"
                                        onClick={resetForm}
                                        className="btn btn-danger"
                                    >
                                        Annuler
                                    </button>
                                </div>
                            </Form>
                        )}
                    </Formik>
                </div>
            </div>
        </div>
    );
};   
export default Register;

\end{verbatim}

\section{Liste des cours}
\begin{verbatim}
import React, { useState, useEffect } from 'react';
import { View, Text, StyleSheet, FlatList, TouchableOpacity, Button } from 'react-native';
import { Link, useRouter } from 'expo-router';
import Icon from 'react-native-vector-icons/MaterialIcons';
import { AuthContext } from '@/context/AuthProvider';

const API_BASE_URL = 'http://localhost:8080/api';
const ITEMS_PER_PAGE = 5;

const ComposantCours = ({navigation}) => {
  const [cours, setCours] = useState([]);
  const [currentPage, setCurrentPage] = useState(1);
  const router = useRouter();
  const { user } = React.useContext(AuthContext);

  const recupererCours = async () => {
    try {
      const response = await fetch(`${API_BASE_URL}/cours`);
      if (!response.ok) throw new Error('Erreur lors de la récupération des cours');
      const data = await response.json();
      setCours(data);
    } catch (error) {
      console.error(error);
    }
  };

    console.log(user);
    
      if (!user) {
          alert("l'utilisateur n'est pas connecter");
          return null;
      }
    
  

  useEffect(() => {
    recupererCours();
   // recupererRoleUtilisateur();  // Récupérer le rôle de l'utilisateur à chaque chargement du composant
  }, []);

  const SupprimerCours = async (id) => {
    if (user.token) {
     
        const response = await fetch(`http://localhost:8080/api/cours/${id}`, {
          method: 'DELETE',
          headers: {
            'Content-Type': 'application/json',
            'Authorization': `Bearer ${user.token}`,
          },
        });
        console.log(response);
        
        
        recupererCours();
     
    } else {
      console.log('Aucune donnée utilisateur trouvée dans localStorage.');
    }
  };

  const paginatedData = () => {
    const startIndex = (currentPage - 1) * ITEMS_PER_PAGE;
    return cours.slice(startIndex, startIndex + ITEMS_PER_PAGE);
  };

  const totalPages = Math.ceil(cours.length / ITEMS_PER_PAGE);

  return (
    <View style={styles.container}>
       <Button
        title="Ajouter un Cours"
        onPress={() => navigation.navigate('addCours')}
                  /> 
      <Text style={styles.title}>Liste des Cours</Text>

      <View style={styles.table}>
        <View style={styles.tableHeader}>
          <Text style={[styles.tableCell, { flex: 2 }]}>Titre</Text>
          <Text style={[styles.tableCell, { flex: 3 }]}>Description</Text>
          <Text style={[styles.tableCell, { flex: 1 }]}>Prix</Text>
          <Text style={[styles.tableCell, { flex: 1 }]}>Crédit</Text>
          <Text style={[styles.tableCell, { flex: 3 }]}>Actions</Text>
        </View>
        <FlatList
          data={paginatedData()}
          keyExtractor={(item) => item.id.toString()}
          renderItem={({ item }) => (
            <View style={styles.tableRow}>
              <Text style={[styles.tableCell, { flex: 2 }]}>{item.titre}</Text>
              <Text style={[styles.tableCell, { flex: 3 }]}>{item.description}</Text>
              <Text style={[styles.tableCell, { flex: 1 }]}>{item.prix} FCfa</Text>
              <Text style={[styles.tableCell, { flex: 1, paddingRight: 10 }]}>{item.credit}</Text>
              <View style={[styles.tableCell, { flex: 3 }]}>
                <View style={styles.actions}>
                  {/* Conditionner l'affichage des boutons 'edit' et 'delete' */}
                  {/* {userRole === 'admin' && ( */}
                    <>
                      < Button
                            title="ed"
                            onPress={() => navigation.navigate('updateCours',{id:item.id,titre:item.titre,description:item.description,prix:item.prix,credit:item.credit})}
                      /> 
                      
                       
                      <TouchableOpacity style={styles.actionButton} activeOpacity={0.7} onPress={() => SupprimerCours(item.id)}>
                        <Icon name="delete" size={20} color="red" />
                      </TouchableOpacity>
                      <Link href={{ pathname: '../quiz/id', params: {  
                      id: item.id,
                      titre: item.titre,
                      description: item.description,
                      prix: item.prix,
                      credit: item.credit,
                    }}}>
                      <TouchableOpacity style={styles.actionButton} activeOpacity={0.7}>
                        <Icon name="quiz" size={20} color="#000" />
                      </TouchableOpacity>
                    </Link>
                    </>
                  {/* )} */}

                 
                  
                  {/* Bouton 'person-add' visible si connecté */}
                  {user.token && (
                    <Link href={{ pathname: '../auth'}}>
                      <TouchableOpacity style={styles.actionButton} activeOpacity={0.7}>
                        <Icon name="person-add" size={20} color="#000" />
                      </TouchableOpacity>
                    </Link>
                  )}
                </View>
              </View>
            </View>
          )}
        />
      </View>

      {/* Pagination Controls */}
      <View style={styles.pagination}>
        <TouchableOpacity onPress={() => setCurrentPage((prev) => Math.max(prev - 1, 1))} disabled={currentPage === 1}>
          <Text style={styles.pageButton}>Précédent</Text>
        </TouchableOpacity>
        <Text style={styles.pageInfo}>{currentPage} / {totalPages}</Text>
        <TouchableOpacity onPress={() => setCurrentPage((prev) => Math.min(prev + 1, totalPages))} disabled={currentPage === totalPages}>
          <Text style={styles.pageButton}>Suivant</Text>
        </TouchableOpacity>
      </View>
    </View>
  );
};

const styles = StyleSheet.create({
  container: {
    flex: 1,
    padding: 20,
    backgroundColor: '#fff',
  },
  title: {
    fontSize: 24,
    fontWeight: 'bold',
    marginBottom: 20,
    textAlign: 'center',
  },
  table: {
    marginTop: 30,
  },
  tableHeader: {
    flexDirection: 'row',
    backgroundColor: '#f1f1f1',
    padding: 10,
    borderRadius: 8,
  },
  tableRow: {
    flexDirection: 'row',
    padding: 20,
    borderBottomWidth: 1,
    borderBottomColor: '#ddd',
  },
  tableCell: {
    flex: 1,
    textAlign: 'center',
    paddingHorizontal: 5,
    minHeight: 50,
  },
  actions: {
    flexDirection: 'row',
    justifyContent: 'space-around',
  },
  actionButton: {
    padding: 10,
    borderRadius: 5,
    alignItems: 'center',
    flexDirection: 'row',
  },
  buttonText: {
    marginLeft: 5,
    fontSize: 14,
  },
  addButton: {
    marginBottom: 10,
    padding: 10,
    backgroundColor: '#007BFF',
    color: '#fff',
    textAlign: 'center',
    borderRadius: 5,
  },
  pagination: {
    flexDirection: 'row',
    justifyContent: 'space-between',
    alignItems: 'center',
    marginTop: 20,
  },
  pageButton: {
    color: '#007BFF',
    fontWeight: 'bold',
  },
  pageInfo: {
    fontSize: 16,
  },
});

export default ComposantCours;

\end{verbatim}
\section{Ajouter un cours}
\begin{verbatim}
import React, { useContext } from 'react';
import { Formik, Field, Form, ErrorMessage } from 'formik';
import * as Yup from 'yup';
import "bootstrap/dist/css/bootstrap.css";
import { useRouter } from 'expo-router';
import { AuthContext } from '../../context/AuthProvider'; // Corrigez le chemin d'importation

const router = useRouter();

const validationSchema = Yup.object().shape({
  titre: Yup.string()
    .min(2, "trop petit")
    .max(255, "trop long!")
    .required("Ce champ est obligatoire"),
  description: Yup.string()
    .min(2, "trop petit")
    .max(255, "trop long!")
    .required("Ce champ est obligatoire"),
  prix: Yup.number()
    .min(0, "le prix doit etre positif")
    .required("le prix est obligatoire"),
  credit: Yup.number()
    .min(0, "le credit doit etre positif")
    .required("le credit est obligatoire"),
});

const initialValues = {
  titre: "",
  description: "",
  prix: "",
  credit: "",
};

const Formulaire = () => {
  const { user } = useContext(AuthContext); // Utilisez le contexte Auth
  if (!user) {
    alert("l'utilisateur n'est pas connecté");
    return null;
  }

  const handleSubmit = async (values) => {
    try {
      const response = await fetch('http://localhost:8080/api/cours', {
        method: 'POST',
        headers: {
          'Content-Type': 'application/json',
          'Authorization': `Bearer ${user.token}`, // Utilisez le token de l'utilisateur
        },
        body: JSON.stringify(values),
      });

      if (!response.ok) {
        throw new Error('Une erreur est survenue lors de l\'enregistrement des données.');
      }

      const data = await response.json();
      console.log('Données enregistrées:', data);
      router.push('cours/liste');
    } catch (error) {
      console.error('Erreur:', error);
    }
  };

  return (
    <div className="container">
      <div className="row">
        <div className="col-md-6 offset-md-3 pt-3">
          <h1 className="text-center">Enregistrer un cours</h1>
          <Formik
            initialValues={initialValues}
            validationSchema={validationSchema}
            onSubmit={handleSubmit}
          >
            {({ resetForm }) => (
              <Form>
                <div className="form-group mb-3">
                  <label htmlFor="titre">Titre : </label>
                  <Field
                    type="text"
                    id="titre"
                    name="titre"
                    className="form-control"
                  />
                  <ErrorMessage
                    name="titre"
                    component="small"
                    className="text-danger"
                  />
                </div>
                <div className="form-group mb-3">
                  <label htmlFor="description">Description : </label>
                  <Field
                    type="text"
                    id="description"
                    name="description"
                    className="form-control"
                  />
                  <ErrorMessage
                    name="description"
                    component="small"
                    className="text-danger"
                  />
                </div>
                <div className="form-group mb-3">
                  <label htmlFor="prix">Prix : </label>
                  <Field
                    type="number"
                    id="prix"
                    name="prix"
                    className="form-control"
                  />
                  <ErrorMessage
                    name="prix"
                    component="small"
                    className="text-danger"
                  />
                </div>
                <div className="form-group mb-3">
                  <label htmlFor="credit"> Credit : </label>
                  <Field
                    type="number"
                    id="credit"
                    name="credit"
                    className="form-control"
                  />
                  <ErrorMessage
                    name="credit"
                    component="small"
                    className="text-danger"
                  />
                </div>
                <div className="form-group d-flex justify-content-end gap-3">
                  <button
                    type="submit"
                    className="btn btn-primary"
                  >
                    Enregistrer
                  </button>
                  <button
                    type="button"
                    onClick={resetForm}
                    className="btn btn-danger"
                  >
                    Annuler
                  </button>
                </div>
              </Form>
            )}
          </Formik>
        </div>
      </div>
    </div>
  );
};

export default Formulaire;
\end{verbatim}
\section{Modifier cours}
\begin{verbatim}
import React from 'react';
import { Formik, Field, Form, ErrorMessage } from 'formik';
import * as Yup from 'yup';
import { Link, useRouter } from 'expo-router';  // Importez useRouter

import { AuthContext } from '@/context/AuthProvider';

import "bootstrap/dist/css/bootstrap.css";

const validationSchema = Yup.object().shape({
    titre: Yup.string()
        .min(2, "trop petit")
        .max(255, "trop long!")
        .required("Ce champ est obligatoire"),
    description: Yup.string()
        .min(2, "trop petit")
        .max(255, "trop long!")
        .required("Ce champ est obligatoire"),
    prix: Yup.number()
        .min(0, "le prix doit etre positif")
        .required("le prix est obligatoire"),
    credit: Yup.number()
        .min(0, "le credit doit etre positif")
        .required("le credit est obligatoire"),
});


interface DetailsProps {
    cours: {
        id:number,

        titre:string,
        description:string,
        prix:number,
        credit:number,
    
    },
}

const ModifierCour = ({route,navigation}) => {
    const initialValues = {
        id: route.params.id,
        titre: route.params.titre,
        description: route.params.description,
        prix: route.params.prix,
        credit: route.params.credit,
    };
    const { user } = React.useContext(AuthContext);
    console.log(user);
    
      if (!user) {
          alert("l'utilisateur n'est pas connecter");
          return null;
      }

    const handleSubmit = async (values) => {
   
        
        try {
            const response = await fetch(`http://localhost:8080/api/cours/${values.id}`, {
                method: 'PUT', // Utilisez PUT pour mettre à jour
                headers: {
                    'Content-Type': 'application/json',
                    'Authorization': `Bearer ${user.token}`,  // Ajouter le token d'authentification

                },
                body: JSON.stringify(values), // Convertit les valeurs en JSON
            });
    
            if (!response.ok) {
                throw new Error('Erreur lors de la mise à jour');
            }
    
            const data = await response.json();
            console.log('Cours mis à jour:', data);
            //router.push('cours/liste');
            navigation.navigate('cours');
    
        } catch (error) {
            console.error('Erreur:', error);
        }
    
 
    };
    return (
        <div className="container">
            <div className="row">
                <div className="col-md-6 offset-md-3 pt-3">
                    <h1 className="text-center">Mettre à jour le cours</h1>
                    <Formik
                        initialValues={initialValues}
                        validationSchema={validationSchema}
                        onSubmit={handleSubmit}
                    >
                        {({ resetForm }) => (
                            <Form>
                                <div className="form-group mb-3">
                                    <label htmlFor="titre">Titre:</label>
                                    <Field
                                        type="text"
                                        id="titre"
                                        name="titre"
                                        className="form-control"
                                    />
                                    <ErrorMessage
                                        name="titre"
                                        component="small"
                                        className="text-danger"
                                    />
                                </div>
                                <div className="form-group mb-3">
                                    <label htmlFor="description">Description:</label>
                                    <Field
                                        type="text"
                                        id="description"
                                        name="description"
                                        className="form-control"
                                    />
                                    <ErrorMessage
                                        name="description"
                                        component="small"
                                        className="text-danger"
                                    />
                                </div>
                                <div className="form-group mb-3">
                                    <label htmlFor="prix">Prix:</label>
                                    <Field
                                        type="number"
                                        id="prix"
                                        name="prix"
                                        className="form-control"
                                    />
                                    <ErrorMessage
                                        name="prix"
                                        component="small"
                                        className="text-danger"
                                    />
                                </div>
                                <div className="form-group mb-3">
                                    <label htmlFor="credit">Credit:</label>
                                    <Field
                                        type="number"
                                        id="credit"
                                        name="credit"
                                        className="form-control"
                                    />
                                    <ErrorMessage
                                        name="credit"
                                        component="small"
                                        className="text-danger"
                                    />
                                </div>

                                <div className="form-group d-flex justify-content-end gap-3">
                                    <button type="submit" className="btn btn-primary">
                                        Mettre à jour
                                    </button>
                                    <button
                                        type="button"
                                        onClick={resetForm}
                                        className="btn btn-danger"
                                    >
                                        Annuler
                                    </button>
                                </div>
                            </Form>
                        )}
                    </Formik>
                </div>
            </div>
        </div>
    );
};
export default ModifierCour;
\end{verbatim}
\section{les navigations}
\begin{verbatim}
import * as React from 'react';
import { Button, Image, StyleSheet, Text, TouchableOpacity, View } from 'react-native';
import ModifierCour from './cours/edit';
import Login from './users/login';
import ComposantCours from './cours';
import Register from './users/register';
import Formulaire from './cours/create';
// Defintion de  L'Écran d'accueil


export function HomeScreen({ navigation}) {

    return (
        <View style={styles.container}>
              <Button
                title="Se Connecter"
                onPress={() => navigation.navigate('login')}
              />              
              <Text>ou</Text>
              <Button
                title="Creer Un Compte"
                onPress={() => navigation.navigate('register')}
              />
            </View>

    );
}



export function loginScreen({ navigation }) {
  return (
    <View style={styles.container}>
     
      <Login navigation={navigation}></Login>
      
    </View>
  );
}
export function registerScreen({ navigation }) {
  return (
    <View style={styles.container}>
     
      <Register navigation={navigation}></Register>
      
    </View>
  );
}




export function listeCours({ navigation }) {
  return (
    <View style={styles.container}>
     
      <ComposantCours navigation={navigation}></ComposantCours>
    </View>
  );
}


export function UpdateCours (props){
  const {navigation,route}=props;
  return ( 
  <View>
     
      <ModifierCour  navigation={navigation} route={route}></ModifierCour>
  </View>);

 
}


// Écran détails
export function CreationCoursScreen({ navigation }) {
    return (
      <View style={styles.container}>
       
       <Formulaire/> 
            
        </View>
    );
  }



  export function deleteCoursScreen({ navigation }) {
    return (
      <View style={styles.container}>
        Supp
      </View>
    );
  }
 


  const styles=StyleSheet.create({
    container: {
       flex: 1,
       backgroundColor: '#fff',
       alignItems: 'center',
       justifyContent: 'center',
    },
    bouton:{
      padding:5,
      borderRadius:20,
      backgroundColor: '#000',
      color:'#fff',
    },
    linkButton: {
      backgroundColor: '#4CAF50', // Couleur de fond verte
      paddingVertical: 12,
      paddingHorizontal: 30,
      borderRadius: 8,
      marginVertical: 10,
      elevation: 5, // Ombre pour effet de surélévation
      shadowColor: '#000', // Ombre
      shadowOffset: { width: 0, height: 2 },
      shadowOpacity: 0.2,
      shadowRadius: 3.5,
    },
    linkText: {
      color: '#fff', // Texte blanc
      fontSize: 16,
      fontWeight: 'bold',
      textAlign: 'center',
    },
  });
\end{verbatim}

\end{document}