\documentclass[5pt]{beamer}

\usepackage[english]{babel}
\usepackage[utf8]{inputenc}
 \usepackage{graphicx}
\usepackage{subcaption}
\usepackage{hyperref}
\usepackage{multicol}
\usepackage{listings}

\usepackage{listings}
\usefonttheme{professionalfonts}
\setbeamertemplate{itemize items}[ball]

\usepackage{enumitem,xcolor}
\definecolor{lightgray}{RGB}{220 220 220}

\lstset{language=C++,tabsize=4, literate={é}{{\'e}}1 {è}{{\`e}}1 {ê}{{\^e}}1 {à}{{\`a}}1, stringstyle=\color{red}, keywordstyle=\color{blue}, identifierstyle=\color{black}}%{É}{{\'E}}1

\newcommand{\code}[1]{\colorbox{lightgray}{\lstinline~#1~}}

\usetheme{Darmstadt}
\definecolor{clearblueifnti}{RGB}{24,116,220}
\setbeamercolor{title}{use=structure,bg=clearblueifnti,fg=white}

\makeatletter
\setbeamertemplate{title page}{
  \vbox{}
  \vfill
  \begingroup
    \centering\hspace{2cm}
    \begin{minipage}{.7\textwidth}
    \begin{beamercolorbox}[sep=8pt,center,shadow=true,rounded=true]{title}
      \usebeamerfont{title}\inserttitle\par%
      \ifx\insertsubtitle\@empty%
      \else%
        \vskip0.25em%
        {\usebeamerfont{subtitle}\usebeamercolor[fg]{subtitle}\insertsubtitle\par}%
      \fi%     
    \end{beamercolorbox}%
    \end{minipage}

    \vskip2.8em\par
    \begin{minipage}{\textwidth}
    \begin{beamercolorbox}[sep=8pt,center]{author}
      \usebeamerfont{author}\insertauthor
    \end{beamercolorbox}
    \end{minipage}

    \begin{minipage}{\textwidth}
    \begin{beamercolorbox}[sep=8pt,center]{institute}
      \usebeamerfont{institute}\insertinstitute
    \end{beamercolorbox}
    \end{minipage}

    \begin{minipage}{\textwidth}
    \begin{beamercolorbox}[sep=8pt,center]{date}
      \usebeamerfont{date}\insertdate
    \end{beamercolorbox}
    \end{minipage}
    \vskip0.5em
    {\usebeamercolor[fg]{titlegraphic}\inserttitlegraphic\par}
  \endgroup
  \vfill
}
\makeatother

\title{UElibre :Présentation de React native }
\institute{IFNTI S4}
\author{Presenté par :SANKARA sarata \& TOURE Chabane}

\date{\today}

\setbeamertemplate{footline}
{
  \hbox{
  \begin{beamercolorbox}[wd=.3\paperwidth,ht=2.25ex,dp=1ex,center]{author in head/foot}
    \usebeamerfont{author in head/foot}\insertshortauthor
  \end{beamercolorbox}
  \begin{beamercolorbox}[wd=.6\paperwidth,ht=2.25ex,dp=1ex,center]{date in head/foot}
    \usebeamerfont{date in head/foot}\insertshortsubtitle{} : \inserttitle{} - \insertshortinstitute{} - \insertdate
  \end{beamercolorbox}
  \begin{beamercolorbox}[wd=.1\paperwidth,ht=2.25ex,dp=1ex,center]{title in head/foot}
    \usebeamerfont{title in head/foot}\insertframenumber{} / \inserttotalframenumber
  \end{beamercolorbox}}
}

\usebackgroundtemplate{\includegraphics[width=\paperwidth,height=\paperheight]{2}}

\begin{document}

{
\setbeamertemplate{headline}{\vskip\headheight}
\setbeamertemplate{footline}{}
\usebackgroundtemplate{\includegraphics[width=\paperwidth,height=\paperheight]{1}}
\begin{frame}[plain]
	%\vspace{5cm}
	\titlepage\end{frame}
}

\begin{frame}{Table des matières}
	%\begin{multicols}{2}
	\tableofcontents
	%\end{multicols}
\end{frame}

{\AtBeginSection[]{
  \begin{frame}
  \vfill
  \centering
  \begin{beamercolorbox}[sep=8pt,center,shadow=true,rounded=true]{title}
    \usebeamerfont{title}\insertsection\par
  \end{beamercolorbox}
  \vfill
  \end{frame}
}



% \section{ Introduction  }





%^\begin{frame}[fragile]{}
%\begin{block}{}
%\begin{center}
%\includegraphics[scale=0.25]{Immo.jpeg}

%\end{center}
%\end{block}
%\end{frame}

% \section{Installation}
%¨\section{Présentation du besoin}

\section{Introduction à React native}
\begin{frame}[fragile]{Introduction}
\begin{block}{Qu'est ce que react vative}
React native est un framework puissant pour créer des applications
mobiles multi-plateformes avec JavaScript et React.Il a été
développer par facebook 
	\end{block}
\end{frame}
\begin{frame}[fragile]{Installation des Outils}
\begin{block}{Installation de l’outil Expo}
$ \bullet $ Expo est un ensemble d’outils et de services conçus pour
faciliter le développement d’applications React Native.

$ \bullet $ Pour installer Expo, utilisez les commandes suivantes:
    \begin{lstlisting}
npm install expo-cli -g (installation globale)
npm install expo-cli (installation locale)
    \end{lstlisting}
	\end{block}
\end{frame}
\section{Création d’un Projet React Native}

\begin{frame}{Avec npx }
    \begin{itemize}
        \item Pour créer un projet React Native, utilisez les commandes suivantes.
        \item npx create-expo-app@latest
        
    \end{itemize}
       
\end{frame}


\section{Rôle des Fichiers et Dossiers du Projet}
\begin{frame}[fragile]{}
\begin{block}{}
Voici une description des principaux fichiers et dossiers générés :\\
$ \bullet $\textbf{le dossier App} : est le coeur de notre application il contient le fichier index.tsx qui est le point d’entrée de l’application.Toutles fichiers qui sont dans ce dossier sont des navigations.

$ \bullet $ \textbf{Le dossier assets }: Contient les images et fichiers statiques.

$ \bullet $  \textbf{node modules} : Contient les dépendances du projet.

 $ \bullet $ \textbf{Le dossier components} : Contient les les composants du
projet.
 $ \bullet $ \textbf{Le dossier constantes} : Contient les constantes du projet.
 
 
 $ \bullet $ \textbf{app.json} : Configuration de l’application.
 $ \bullet $ \textbf{ package.json }: Liste des dépendances et configurations.
 
 $ \bullet $\textbf{ Le dossier script} : contient les scripts de notre projet
 $ \bullet $\textbf{ babel.config.js }: Configuration de Babel pour transpiler le
code.
 
 $ \bullet $ \textbf{package-lock.json} : Assure la cohérence des versions des
dépendances.
    \end{block}
\end{frame}

\begin{frame}[fragile]{Démarrage et Exécution du Projet}
\begin{block}{}
Commandes pour démarrer le projet :
\begin{lstlisting}
npm run start : Démarre le projet React Native.
npm run android : Démarre le projet sur Android.
npm run web : Démarre le projet sur le web
 \end{lstlisting}
    \end{block}
\end{frame}
















\section{Composants }
\begin{frame}[fragile]{Composant}
\begin{block}{Création de composants}
Un composant est une fonction ou une classe qui retourne du JSX.\\
Exemple :
\begin{lstlisting}
function Bonjour(props) {
  return <h1>Bonjour, {props.nom}!</h1>;
}
\end{lstlisting}
    \end{block}
\end{frame}

\begin{frame}[fragile]{Props}
\begin{block}{Utilisation des Props}
Les \textbf{props} sont des arguments passés aux composants pour les personnaliser. Exemple :
\begin{lstlisting}
<Bonjour nom="Jean" />
\end{lstlisting}
    \end{block}
\end{frame}

\begin{frame}[fragile]{Composants native}
\begin{block}{Qu'est ce qu'un composant}
Un composant en React Native est une unité de code représentant
une partie de l’interface utilisateur.
Quelques composants natives sont :
   \end{block}
\end{frame}

\begin{frame}[fragile]{Composant Text}
\begin{block}{}
\textbf{Le composant Text} sert à afficher des chaines de texte et gère les événements tactiles.\\
Fourni par React Native, il est analogue à la balise <p> en
HTML.
\begin{verbatim}
import React from 'react';
import {  Text,View } from 'react-native';
export default function Home() {
  return ( 
<View>
	<Text> Bienvenue sur mon projet react native</Text>
</View>)
\end{verbatim}

    \end{block}
\end{frame}

\begin{frame}[fragile]{Composant Image}
\begin{block}{}
\textbf{Le composant Image}  sert à afficher différents types
d’images.L’attribut source prend uri, qui contient le chemin de
l’image, et on peut avoir l’attribut style pour personnaliser
l’affichage de l’image.
\begin{verbatim}
import React from 'react';
import {  Image } from 'react-native';
export default function Home() {
  return (
<Image style={styles.image} source={{ uri: ’https://
reactnative.dev/img/tiny_logo.png’ }} />
)
\end{verbatim}


    \end{block}
\end{frame}
\begin{frame}[fragile]{Composant Link}
\begin{block}{}
\textbf{Le composant Link}  est utilisé pour naviguer entre les écrans
\begin{verbatim}
import React from 'react';
import {  Text } from 'react-native';
import { Link } from 'expo-router'; 
export default function Home() {
  return ( 
<Link href="/propos">
	<Text>a propos</Text>
</Link>)
\end{verbatim}


    \end{block}
\end{frame}

\begin{frame}[fragile]{Composant View}
\begin{block}{}
\textbf{Le composant View}\\
Conteneur prenant en charge la mise en page, le style et les
contrôles d’accessibilité.\\
Analogue à la balise <div> en HTML
\begin{verbatim}
import React from 'react';
import { View, Text } from 'react-native';
export default function Home() {
  return (
<View>
	<Text>bienvenue</Text>
</View>
);
\end{verbatim}
\end{block}
\end{frame}

\begin{frame}[fragile]{Composant Stack}
\begin{block}{}
\textbf{Le composant Stack}\\
permer de créer une pile de navigation et
permet d’ajouter de nouveaux itinéraires dans votre application
\begin{verbatim}
import { Stack} from 'expo-router';
import { View } from 'react-native';


    return (
         <View style={styles.container}>
              <Stack.Screen
                options={{
                  title: "page de login",
                }}
              />
             
            </View>
    )
}
\end{verbatim}
\end{block}
\end{frame}

\begin{frame}[fragile]{Composant StyleSheet}
\begin{block}{}
\textbf{Le composant StyleSheet}\\
Module utilisé pour créer et gérer les styles.\\
 Avantages :\\
 Organisation : Les styles sont centralisés.\\
 Validation : Les propriétés des styles sont vérifiées.\\
 Réutilisabilité : Les styles peuvent être appliqués à plusieurs
composants.\\
 Clarté : Les styles sont séparés de la logique.
 Les noms des propriétés suivent le format camelCase (ex.
backgroundColor)
\begin{verbatim}

import  React from 'react';
import {  StyleSheet, Text, View } from 'react-native';


export default function Index() {
   return (

    <View style={styles.container} >
      	<Text> Test StyleSheet </Text>
    </View>
    
   );
 }


 const styles=StyleSheet.create({
  container: {
     flex: 1,
     backgroundColor: '#fff',
     alignItems: 'center',
     justifyContent: 'center',
  },
  
});
\end{verbatim}
\end{block}
\end{frame}


\section{les composants en react}

\begin{frame}[fragile]{}
\begin{block}{}

Nous avons deux types de composants les composants fonctionnels
et les composants de classe

\end{block}
\end{frame}

\begin{frame}[fragile]{Les composants fonctionnel}
\begin{block}{}

ce sont des fonctions qui retourne du javaScript Xml (JSX) ou TypeScript xml (TSX).Exemple
\begin{verbatim}
import  React from 'react';
import {  StyleSheet, Text, View } from 'react-native';


export default function ComposantFonctionel() {
   return (

    <View style={styles.container} >
      	<Text> Test ComposantFonctionel </Text>
    </View>
    
   );
 }


 const styles=StyleSheet.create({
  container: {
     flex: 1,
     backgroundColor: '#fff',
     alignItems: 'center',
     justifyContent: 'center',
  },
  
});
\end{verbatim}
\end{block}
\end{frame}


\begin{frame}[fragile]{Les composants de classe}
\begin{block}{}

ce sont des classe qui heritent de la classe Component.Exemple
\begin{verbatim}
import React, { Component } from 'react';

class Compteur extends Component {
  constructor(props) {
    super(props);
    this.state = { count: 0 }; // Déclaration du state
  }

  incrementer = () => {
    this.setState({ count: this.state.count + 1 }); // Modification du state
  };

  render() {
    return (
      <div>
        <h1>Compteur : {this.state.count}</h1>
        <button onClick={this.incrementer}>Incrémenter</button>
      </div>
    );
  }
}

export default Compteur;
\end{verbatim}
\end{block}
\end{frame}





\section{Les props et state}
\begin{frame}[fragile]{ Les props}
\begin{block}{Les props }
Les props : Transmettent des informations à un composant
\end{block}
\end{frame}
\begin{frame}[fragile]{ Les state}
\begin{block}{Les state }
Le state : Variable interne qui existe à l’intérieur d’un
composant

\end{block}
\end{frame}
\begin{frame}[fragile]{ Différences entre props et state}
\begin{block}{Props }
Données immuables.\\
Transmises d’un composant à un autre.
\end{block}
\begin{block}{State }
Données mutables.\\
Utilisées uniquement à l’intérieur du composant
\end{block}
\end{frame}

\section{Les providers}
\begin{frame}[fragile]{ Définition}
\begin{block}{}
Provider permet de transmettre les données du context aux composants enfants.

\end{block}
\end{frame}
\begin{frame}[fragile]{ createContext}
\begin{block}{createContext }
createContext permet de créer un contexte que les composants peuvent fournir ou lire\\
const SomeContext = createContext(defaultValue)

\end{block}
\end{frame}
\begin{frame}[fragile]{Exemple de provider pour l'authentification}
\begin{block}{ }

\begin{verbatim}
import React, { createContext, useState,
 useEffect, ReactNode } from 'react';
const initialValues = {
    email: "",
    token: "",
    authorities: "",
  };
export const AuthContext = createContext();
export const AuthProvider = ({ children }) => {
  const [user, setUser] = useState(initialValues);
  \end{verbatim}

\end{block}
\end{frame}
\begin{frame}[fragile]{}
\begin{block}{ }
\begin{verbatim}

  return (
    <AuthContext.Provider value={{ user, setUser }}>
      {children}
    </AuthContext.Provider>
  );
};

\end{verbatim}
\end{block}

\end{frame}
\begin{frame}[fragile]{useContext}
\begin{block}{useContext }
useContext permet de aux composant enfants d’accéder au context depuis le composant parent
\end{block}
\end{frame}

\begin{frame}[fragile]{Exemple}
\begin{block}{utiliser}

\begin{verbatim}
import React, { useEffect } from 'react';
import {Formik, Field, Form, ErrorMessage} from 'formik';
import * as Yup from 'yup';
import "bootstrap/dist/css/bootstrap.css";
import { AuthContext } from '@/context/AuthProvider';
const validationSchema = Yup.object().shape({
    email: Yup.string()
        .email("email invalide")
        .required("l'email est obligatoire"),
    password: Yup.string()
        .required("Mot de passe est obligatoire")
        .min(4, "Mot de passe doit être plus grand que 8 caractères")
        .max(50, "Mot de passe doit être plus petit que 50 caractères"),
});
\end{verbatim}

\end{block}
\end{frame}

\begin{frame}[fragile]{}
\begin{block}{}
\begin{verbatim}

interface DetailsProps {
    cours: {
        id:number,
    },
  }

  const Login =  ({ navigation }) => {
    const initialValues = {
        email: "",
        password: "",
    };

    const { user , setUser } = React.useContext(AuthContext);
  
const handleSubmit = async (values) => { 
\end{verbatim}

\end{block}
\end{frame}

\begin{frame}[fragile]{}
\begin{block}{}
\begin{verbatim}
   
    try {
        const response = await fetch('http://localhost:8080/api/login', {
            method: 'POST',
            headers: {
                'Content-Type': 'application/json',
            },
            body: JSON.stringify(values),
        });

        if (!response.ok) {
          //  throw new Error('Une erreur est survenue lors de l\'enregistrement des données.');
            alert("mot de passe ou email incorrect");
        }

        const data = await response.json();
        console.log('Données enregistrées:', data);
        \end{verbatim}

\end{block}
\end{frame}

\begin{frame}[fragile]{}
\begin{block}{}
\begin{verbatim}
        setUser(data);
        if(user){
            navigation.navigate('cours');
        }
    }
    
    catch (error) {
        console.error('Erreur:', error);
        // Optionally handle errors appropriately, e.g., show a notification
    }
}

useEffect (() => {  
    console.log(user);
    
}, [user]);
\end{verbatim}

\end{block}
\end{frame}

\begin{frame}[fragile]{}
\begin{block}{}
\begin{verbatim}
    return (
        <div className="container">
            <div className="row">
                <div className="col-md-6 offset-md-3 pt-3">
                    <Formik
                        initialValues={initialValues}
                        validationSchema={validationSchema}
                        onSubmit={handleSubmit}
                        >
                        {({ resetForm }) => (
                            <Form>
                                
                                <div className="form-group mb-3">
                                    <label htmlFor="email">
                                        email:
                                    </label>
                                    \end{verbatim}

\end{block}
\end{frame}

\begin{frame}[fragile]{}
\begin{block}{}
\begin{verbatim}
                                    <Field
                                        type="email"
                                        id="email"
                                        name="email"
                                        className="form-control"
                                    />
                                    <ErrorMessage
                                        name="email"
                                        component="small"
                                        className="text-danger"
                                    />
                                </div>
                              
                                <div className="form-group mb-3">
                                    <label htmlFor="password">
                                        Mot de passe:
                                    </label>
                                    \end{verbatim}

\end{block}
\end{frame}

\begin{frame}[fragile]{}
\begin{block}{}
\begin{verbatim}
                                    <Field
                                        type="password"
                                        id="password"
                                        name="password"
                                        className="form-control"
                                    />
                                    <ErrorMessage
                                        name="password"
                                        component="small"
                                        className="text-danger"
                                    />
                                </div>
                               
                               
                                <div className="form-group d-flex justify-content-end gap-3">
                                    <button
                                        type="submit"
          \end{verbatim}

\end{block}
\end{frame}

\begin{frame}[fragile]{}
\begin{block}{}
\begin{verbatim}
                                        className="btn btn-primary"
                                    >
                                        Se connecter
                                    </button>
                                    <button
                                        type="button"
                                        onClick={resetForm}
                                        className="btn btn-danger"
                                    >
                                        Annuler
                                    </button>
                                </div>
                            </Form>
                        )}
                    </Formik>
                </div>
            </div>
            \end{verbatim}

\end{block}
\end{frame}

\begin{frame}[fragile]{}
\begin{block}{}
\begin{verbatim}
        </div>
    );
};   
export default Login;

\end{verbatim}

\end{block}

\end{frame}



\section{Les Navigations}
\begin{frame}[fragile]{ }
\begin{block} {}
La navigation est une bibliotheque qui permet de configurer les
écrans dune application\end{block}
\end{frame}
\begin{frame}[fragile]{ Installation et mise en place}
\begin{block}{}
\begin{verbatim}
npm install @react-navigation/native 
@react-navigation/native-stack

npx expo install react-native-screens
react-native-safe-area-context
\end{verbatim}
\end{block}
\end{frame}
\begin{frame}[fragile]{ Exemple d'utilisation}
\begin{block}{ }

\end{block}

\end{frame}


\section{Gestion de l’état et des données}
\begin{frame}[fragile]{ Gestion de l’état avec React}
\begin{block} {Utiliser le state local et les hooks (useState, useEffect)}
le hook useState est utilisé pour déclarer des states dans un
composant fonctionnel
Syntaxe
\begin{verbatim}
const [state, setState] = useState(initialValue) ;

\end{verbatim}
\end{block}
\end{frame}

\begin{frame}[fragile]{ }
\begin{block} {}
Exemple :

\begin{verbatim}
import React, { useState } from ’react’;
function Compteur() {
// Déclare un état "count" avec une valeur initiale de 0
const [count, setCount] = useState(0);
return (
<div>
<p>Compteur : {count}</p>
<button onClick={() => setCount(count + 1)}>
Incrémenter</button>
</div>
);
}
\end{verbatim}

\end{block}
\end{frame}
\begin{frame}[fragile]{}
\begin{block}{}
useEffect est utilisé pour gérer les effets secondaires dans un
composant fonctionnel. Un effet secondaire peut être toute
opération qui a un impact en dehors du composant, comme la
récupération de données depuis une API,syntaxe
\begin{verbatim}
useEffect(() => {
}, [dependencies]);
\end{verbatim}
\end{block}
\end{frame}
\begin{frame}[fragile]{ Exemple d'utilisation}
\begin{block}{ }

\end{block}

\end{frame}
\begin{frame}[fragile]{Appels API et gestion des données}
\begin{block}{ntroduction à Fetch API pour récupérer des données}
fecth est une méthode qui permet de contacter l’endpoint d’ uneAPI et retourne une promesse
Exemple :récupérer les cours depuis api de spring boot
\begin{verbatim}
const API_BASE_URL = 'http://localhost:8080/api';
const recupererCours = async () => {
try {
const response = await fetch(‘${API_BASE_URL}
/cours‘);
if (!response.ok) throw new
Error(’Erreur lors de la récupération des cours’);
const data = await response.json();
setCours(data);
} catch (error) {
console.error(error);
}
};
\end{verbatim}
\end{block}
\end{frame}
\begin{frame}[fragile]{ Afficher des données dynamiques dans l’application}
\begin{block}{ }
Pour afficher les données dynamiques nous allons utiliser le
composant \textbf{FlatList}.Il a trois propriétés importantes\\
\textbf{data} : La source des données (le tableau des éléments à
afficher).\\
\textbf{keyExtractor} : Fournit une clé unique pour chaque élément.\\
\textbf{renderItem} : Détermine comment chaque élément doit être
affiché.\\
Exemple :affichons les cours récupérer de manière dynamique
\begin{verbatim}
<FlatList
data={recupererCours}
keyExtractor={(item) => item.id.toString()}
renderItem={({ item }) => (
<View style={styles.tableRow}>
<Text>{item.titre}</Text>
<Text>{item.description}</Text>
<Text>{item.prix} FCfa</Text>
<Text}>{item.credit}</Text>
)}
/>
\end{verbatim}
\end{block}

\end{frame}
\section{Formulaires}
\begin{frame}[fragile]{Utiliser Formik et Yup pour la validation des formulaires}
\begin{block}{Yup }
Yup est une bibliothèque de validation d’objets pour JavaScript qui vous permet de définir des règles de validation pour des données,
généralement dans le cadre de la gestion de formulaires.IL offre
une variété de méthodes pour valider différents types de données \\
Exemple
\begin{verbatim}
Yup.string() : Pour valider des chaines de caractères.
Yup.number() : Pour valider des nombres.
Yup.boolean() : Pour valider des valeurs booléennes (true ou false).
Yup.date() : Pour valider des objets de type Date.
Yup.array() : Pour valider des tableaux.
Yup.object() : Pour valider des objets.
\end{verbatim}
\end{block}
\end{frame}
\begin{block}{Formik }
Le composant Formik est utilisé pour gérer l’état du formulaire. Il prend plusieurs propriétés :\\
\textbf{initialValues} : Il définit les valeurs initiales des champs du
formulaire.\\
\textbf{validationSchema} :qui vérifie si les données sont valide\\
\textbf{onSubmit} : C’est la fonction qui sera appelée lorsque le formulaire est soumis.\\
\end{block}

\begin{verbatim}
import React, { useContext } from 'react';
import { Formik, Field, Form, ErrorMessage } from 'formik';
import * as Yup from 'yup';
import "bootstrap/dist/css/bootstrap.css";
import { useRouter } from 'expo-router';
import { AuthContext } from '../../context/AuthProvider'; // Corrigez le chemin d'importation

const router = useRouter();

const validationSchema = Yup.object().shape({
  titre: Yup.string()
    .min(2, "trop petit")
    .max(255, "trop long!")
    .required("Ce champ est obligatoire"),
  description: Yup.string()
  
    .min(2, "trop petit")
    .max(255, "trop long!")
    .required("Ce champ est obligatoire"),
  prix: Yup.number()
    .min(0, "le prix doit etre positif")
    .required("le prix est obligatoire"),
  credit: Yup.number()
    .min(0, "le credit doit etre positif")
    .required("le credit est obligatoire"),
});

const initialValues = {
  titre: "",
  description: "",
  prix: "",
  credit: "",
};

const Formulaire = () => {
  const { user } = useContext(AuthContext); // Utilisez le contexte Auth
  if (!user) {
    alert("l'utilisateur n'est pas connecté");
    return null;
  }

  const handleSubmit = async (values) => {
    try {
      const response = await fetch('http://localhost:8080/api/cours', {
        method: 'POST',
        headers: {
          'Content-Type': 'application/json',
          'Authorization': `Bearer ${user.token}`, // Utilisez le token de l'utilisateur
        },
        body: JSON.stringify(values),
      });

      if (!response.ok) {
        throw new Error('Une erreur est survenue lors de l\'enregistrement des données.');
      }

      const data = await response.json();
      console.log('Données enregistrées:', data);
      router.push('cours/liste');
    } catch (error) {
      console.error('Erreur:', error);
    }
  };

  return (
    <div className="container">
      <div className="row">
        <div className="col-md-6 offset-md-3 pt-3">
          <h1 className="text-center">Enregistrer un cours</h1>
          <Formik
            initialValues={initialValues}
            validationSchema={validationSchema}
            onSubmit={handleSubmit}
          >
            {({ resetForm }) => (
              <Form>
                <div className="form-group mb-3">
                  <label htmlFor="titre">Titre : </label>
                  <Field
                    type="text"
                    id="titre"
                    name="titre"
                    className="form-control"
                  />
                  <ErrorMessage
                    name="titre"
                    component="small"
                    className="text-danger"
                  />
                </div>
                <div className="form-group mb-3">
                  <label htmlFor="description">Description : </label>
                  <Field
                    type="text"
                    id="description"
                    name="description"
                    className="form-control"
                  />
                  <ErrorMessage
                    name="description"
                    component="small"
                    className="text-danger"
                  />
                </div>
                <div className="form-group mb-3">
                  <label htmlFor="prix">Prix : </label>
                  <Field
                    type="number"
                    id="prix"
                    name="prix"
                    className="form-control"
                  />
                  <ErrorMessage
                    name="prix"
                    component="small"
                    className="text-danger"
                  />
                </div>
                <div className="form-group mb-3">
                  <label htmlFor="credit"> Credit : </label>
                  <Field
                    type="number"
                    id="credit"
                    name="credit"
                    className="form-control"
                  />
                  <ErrorMessage
                    name="credit"
                    component="small"
                    className="text-danger"
                  />
                </div>
                <div className="form-group d-flex justify-content-end gap-3">
                  <button
                    type="submit"
                    className="btn btn-primary"
                  >
                    Enregistrer
                  </button>
                  <button
                    type="button"
                    onClick={resetForm}
                    className="btn btn-danger"
                  >
                    Annuler
                  </button>
                </div>
              </Form>
            )}
          </Formik>
        </div>
      </div>
    </div>
  );
};

export default Formulaire;
\end{verbatim}
\end{frame}


\section{Authentification avec provider}

\section{Le deploiement }































\section{Conclusion}
\begin{frame}[fragile]{Conclusion}
Avec ces bases, vous pouvez commencer à créer des interfaces
utilisateur simples et efficaces dans React Native.
\end{frame}


\section{MERCI POUR VOTRE ATTENTION}




\end{document}
